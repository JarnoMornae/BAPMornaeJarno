\chapter{\IfLanguageName{dutch}{Stand van zaken}{State of the art}}%
\label{ch:stand-van-zaken}

% Tip: Begin elk hoofdstuk met een paragraaf inleiding die beschrijft hoe
% dit hoofdstuk past binnen het geheel van de bachelorproef. Geef in het
% bijzonder aan wat de link is met het vorige en volgende hoofdstuk.

% Pas na deze inleidende paragraaf komt de eerste sectiehoofding.

Dit hoofdstuk bevat je literatuurstudie. De inhoud gaat verder op de inleiding, maar zal het onderwerp van de bachelorproef *diepgaand* uitspitten. De bedoeling is dat de lezer na lezing van dit hoofdstuk helemaal op de hoogte is van de huidige stand van zaken (state-of-the-art) in het onderzoeksdomein. Iemand die niet vertrouwd is met het onderwerp, weet nu voldoende om de rest van het verhaal te kunnen volgen, zonder dat die er nog andere informatie moet over opzoeken \autocite{Pollefliet2011}.

Je verwijst bij elke bewering die je doet, vakterm die je introduceert, enz.\ naar je bronnen. In \LaTeX{} kan dat met het commando \texttt{$\backslash${textcite\{\}}} of \texttt{$\backslash${autocite\{\}}}. Als argument van het commando geef je de ``sleutel'' van een ``record'' in een bibliografische databank in het Bib\LaTeX{}-formaat (een tekstbestand). Als je expliciet naar de auteur verwijst in de zin (narratieve referentie), gebruik je \texttt{$\backslash${}textcite\{\}}. Soms is de auteursnaam niet expliciet een onderdeel van de zin, dan gebruik je \texttt{$\backslash${}autocite\{\}} (referentie tussen haakjes). Dit gebruik je bv.~bij een citaat, of om in het bijschrift van een overgenomen afbeelding, broncode, tabel, enz. te verwijzen naar de bron. In de volgende paragraaf een voorbeeld van elk.

\textcite{Knuth1998} schreef een van de standaardwerken over sorteer- en zoekalgoritmen. Experten zijn het erover eens dat cloud computing een interessante opportuniteit vormen, zowel voor gebruikers als voor dienstverleners op vlak van informatietechnologie~\autocite{Creeger2009}.

Let er ook op: het \texttt{cite}-commando voor de punt, dus binnen de zin. Je verwijst meteen naar een bron in de eerste zin die erop gebaseerd is, dus niet pas op het einde van een paragraaf.

\tcbsubtitle{Introductie tot chatbots.}
Een chatbot is software die een menselijke conversatie simuleert. De interactie tussen het softwarepakket en de persoon gebeurt aan de hand van tekstinvoer of spraak. Het is een service aangedreven door regels en soms artificiële intelligentie ~\autocite{Schlicht2016}.

\textbf{Opbouw van een chatbot}\newline
De opbouw van het programma ziet er hetzelfde uit als een normale applicatie. Er is een applicatielaag, een database en een user interface waarmee de gebruiker in contact komt.  In de database laag bevinden zich gesprekken, documenten en informatie die de applicatielaag gaat gebruiken om antwoorden te vormen en de gebruiker verder te helpen. Deze basis opbouw vindt men in elke chatbot terug ongeacht de complexiteit. ~\autocite{Techlabs2017} \newline

Chatbots kunnen opgedeeld worden in 2 grote groepen op basis van complexiteit. In de eerste groep bevinden zich de chatbots die gebaseerd zijn op een set regels. In de tweede groep bevinden zich de meer complexe chatbots die gebruik maken van machine learning. De chatbots die gebaseerd zijn op regels, zijn gelimiteerd. Ze kunnen enkel antwoorden op specifieke commando’s. Als het commando niet correct is, weet het programma niet wat het moet doen. Chatbots die technieken gebruiken vanuit machine learning zijn complexer, maar ook intelligenter. Deze bot heeft een artificieel brein, beter bekend als artificiële intelligentie. De bot “begrijpt” de taal (invoer) en niet enkel het commando. Complexiteit is de belangrijkste factor bij het onderverdelen van soorten chatbots. ~\autocite{Schlicht2016}. \newline

Chatbots die gebruik maken van artificiële intelligentie worden ingedeeld in 3 classificatiemethoden met elk hun eigen complexiteit. De makkelijkste methode kunnen we als volgt benoemen: “vergelijken van patronen”. De applicatielaag gaat de input van de gebruiker vergelijken met de data die teruggevonden wordt in de database. Als de applicatielaag vergelijkende patronen vindt in de database, kiest het programma het meest geschikte antwoord. Hieronder bevindt  zich een beschrijving over de werkwijze van het programma ~\autocite{Techlabs2017}. \newline
Een gebruiker vraagt: Wie is Abraham Lincoln? De applicatie gaat op zoek naar overeenkomstige patronen in de database en vindt de naam Abraham Lincoln terug. Alleen werd de vraag in deze data als volgt geformuleerd: 'Weet jij toevallig wie Abraham Lincoln was?'. Het programma zal het antwoord nemen van de vraag die reeds beantwoord werd, aangezien er een overeenkomst is van het patroon Abraham Lincoln. 
Bovenstaande werking kan nog beter gemaakt worden. Door er meer features aan toe te voegen en complexere code te gaan implementeren, kan het programma in een complexere classificatiemethode terecht komen. \newline

De volgende classificatiemethode is iets complexer en meer gesofisticeerd dan de methode die patronen vergelijkt. Het basisprincipe is nog steeds hetzelfde. De input wordt vergeleken met informatie uit de database. Op dit niveau gaat de applicatie de data en de input vergelijken aan de hand van een algoritme. Voor elke input moet er een uniek patroon te vinden zijn in de database dat voor een gepast antwoord zorgt. 
Hierdoor worden de antwoorden in classificaties beperkt en krijgt de data meer structuur. Een voorbeeld van een algoritme om natural language processing toe te passen, is het algoritme van Bayes. Dit algoritme gaat string inputs vergelijken en een score toekennen. De klasse die het meest overeenkomt met de input, wordt gekozen. Deze classificatiemethode is reeds complexer en meer ontwikkeld dan de vorige classificatiemethode. \newline 
De laatste classificatie en de volgende stap in complexiteit zijn de artificiële neurale netwerken. Hierbij gaat de input string gebruik maken van gewogen connecties die gevormd werden toen de data getraind werd. Door deze training geeft het programma meer accurate antwoorden. Iedere keer dat de data getraind wordt, gaat het algoritme het gewicht van de connectie verbeteren. Op deze manier worden de connecties steeds beter en de antwoorden accurater. 
Hoe het programma omgaat met de gegeven input van de gebruiker wordt natural language processing genoemd ~\autocite{Techlabs2017}.

\tcbsubtitle{Technologie binnen het onderwijs}
Technologie binnen het onderwijs heeft betrekking tot het gedisciplineerd toepassen van kennis met als doel het leren, de instructie en de prestatie te verbeteren. 
Gedisciplineerd toepassen van kennis werd in de definitie opgenomen zodat de definitie weergeeft dat professionals die bezig zijn meet technologie binnen het onderwijs zich baseren op emperisch onderzoek, ervaringen en theorieën. 
Componenten die het leren en de instructies ondersteunen, worden op een hiërarchische manier voorgesteld. Aan de basis van de hiërarchie staan de informatie objecten van data, feiten, discussies, video's en andere bronnen terug. In klasse 2 zijn de gevalideerde en bevestigde kennisobjecten terug te vinden. Klasse 3 staat voor de leerobjecten die gelinkt werden aan een leerdoel. Klasse 4 zijn de instructie objecten. Deze klasse bevat de leerobjecten met feedback, activiteiten en taken. Klasse 5 zijn de lessen. Een gestructureerde verzameling van instructieobjecten. Binnen klasse 6 bevinden zich de programma's. Programma's zijn een gestructureerde verzameling van verschillende lessen. Tot slot sluit het levenslang leren de hiërarchie af. 
Technologieën binnen het onderwijs kunnen voorkomen op elk niveau maar zijn vooral belangrijk bij klasse 4 aangezien hier vooral de ondersteunde rol naar boven komt. ~\autocite{Spector2022}
Karakterstieken van technologieën binnen het onderwijs kunnen gecategoriseerd worden onder 3 pijlers. 
1 Nodige eigenschappen
2 wenselijke eigenschappen
3 waarschijnlijke eigenschappen
1 Een Technologie binnen het onderwijs moet effectief, efficiënt en schaalbaar zijn. De effectiviteit kan gemeten worden aan de hand van de vooruitgang van de kinderen. Een efficiënte tool valideert de info van de student en zorgt voor de nodige feedback. Tot slot is het de bedoeling dat de technologie makkelijk op grote schaal geïmplementeerd kan worden en in verschillende contexten kan ingezet worden. 
2 Wenselijke eigenschappen van een tool binnen het onderwijs zijn: aantrekkelijk, flexibel, aanpasbaar en personaliseerbaar. Een spel dat gelinkt is aan een leerdoel kan de technologie aantrekkelijk maken. Een automatische herconfiguratie afhankelijk van de omgeving is een bijvoorbeeld een wenselijke flexibele feature. Een tool die op basis van het user profiel aanpast en die bijvoorbeeld rekening houdt met bepaalde historieken is en aanpasbare personaliseerbare tool. Tot slot zijn conversaties, reflecties en inovatie waarschijnlijke eigenschappen van de technologie. Denk maar aan NLP, promts die aanzetten tot reflectie en ontwikkeling van nieuwe technologieën ter ondersteuning van huidige leerprocessen. ~\autocite{Spector2022}

\tcbsubtitle{Voordelen van chatbots binnen het onderwijs}
De Beijing Language and Culture University deed onderzoek naar het effect van chatbot-assisted learning. 
Uit de studie kwam een verbetering naar voor op vlak van leerprestaties, expliciet redeneren en het vasthouden van de kennis. 
De integratie van chatbots biedt voordelen aan zoals onmiddellijke hulp, snelle toegang tot informatie, verbeterde leerresultaten en een betere onderwijsomgeving. 
De bevindingen met betrekking tot kritisch denken, leerbetrokkenheid en motivatie waren tegenstrijdig. 
Chatbots hadden een positief effect op het aanleren van nieuwe zaken maar niet op de motivatie van de leerlingen ~\autocite{Deng2023}. 
Het gebruik van chatbots in de klas zorgt voor minder stres, meer motivatie en zorgt ervoor dat leerlingen op hun eigen tempo kunnen bijleren. 
De meeste leerlingen zijn tevreden met het gebruik van een chatbot maar ondervinden wel problemen op taal-vlak. ~\autocite{AitBaha2023}
https://www.researchgate.net/profile/Augustine-Ifelebuegu/publication/374169179_Chatbots_and_AI_in_Education_AIEd_tools_The_good_the_bad_and_the_ugly/links/65127d3bcce2460b6c35711d/Chatbots-and-AI-in-Education-AIEd-tools-The-good-the-bad-and-the-ugly.pdf todo

\tcbsubtitle{Gebruik en rollen van chatbots}
Het merendeel van reeds bestaande chatbots gaan aan de slag als `teaching agents` en/of als `peer agents`. Reeds bestaande chatbots fungeren dus als hulpleerkracht of als hulp op maat van de student. De chatbot zal bijvoorbeeld tutorials aanraden bij het verwerken van bepaalde leerstof of vragen gaan beantwoorden die de gebruiker aan het systeem stelt. Een klein aantal chatbots gaat te werk als `teachable agent` en als `motivational agent`. Zo zijn er bijvoorbeeld chatbots die de gebruiker een mathematisch probleem helpen op te lossen door het probleem op te splitsen in kleinere problemen en zo een stappenplan opstellen voor de gebruiker. Enkele chatbots proberen ook gevoel aan te brengen in hun antwoorden om leerlingen te motiveren. 
Chatbots worden vooral gebruikt voor leren op maat. Daarnaast worden ze ook gebruikt voor experimenteel leren, sociale dialoog, colaboratief leren, doelgericht leren en leren door onderwijs. 

